\documentclass{compte-rendu} % ou dm, rapport, etc.
\usepackage{lipsum}

\titre{4\ieme{} conférence de Goubaulogie}

\begin{document}
\section{Test}
\lipsum[1-2]

\begin{citation}[Jean Goubault-Larrecq]
Je sens la pression que je devrais être quelqu'un d'essence divine que je ne suis pas.
\end{citation}

\begin{question}
\begin{enumerate}
    \item Montrer que, pour toute matrice \(A\in\A_n(\Rr)\), \(\exp(A)\in\SO_n(\Rr)\).
    \item Soit $S\in\M$, telle que: \(\forall U\in\ort_n(\Rr), \tr(SU) \leq \tr(S)\). Montrer que \(S\in \S_n^{+}(\Rr)\).
    \item Montrer alors que, pour toute matrice \(A\in\M\):
\[
    \exists (\Omega, S) \in \ort_n(\Rr)\times \S_n^+(\Rr), A = S \Omega
.\]
    Montrer de plus que, pour $A\in\GL_n(\Rr)$:
\[
    \exists! (\Omega, S) \in \ort_n(\Rr)\times \S_n^{++}(\Rr), A = S \Omega
.\] 
\end{enumerate}
\end{question}

\begin{enumerate}
\item \lipsum[17]

\begin{figure}[ht]
    \centering
    \includegraphics[width=0.5\textwidth]{example-image-duck}
    \caption{Never forgetti}
\end{figure}
\item \lipsum[18]
\end{enumerate}
\end{document}
